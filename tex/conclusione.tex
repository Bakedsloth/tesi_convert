\section{motivazioni}
Una parte della bioinformatica si occupa di ricostruire gli alberi evoluzionari di eventi di mutazione che si suppone abbiano creato un tumore, usando diversi modelli per inferire l' albero evoluzionale da un campione di tumore. Tuttavia, la grandezza e risoluzione dei dataset con cui abbiamo a che fare sono in continuo aumento, portando allo sviluppo di celluloid\cite{patterson}, un nuovo metodo per ridurre la grandezza dei dataset raggruppando  delle mutazioni. Il formato di input ed output di celluloid è lo stesso del formato di input di SASC \cite{sasc}, ma si voleva rendere facilmente fruibile celluloid anche ad altri metodi che utilizzano dataset di sequenziamento a singola cellula.

\section{conclusioni}
Lo strumento \textit{convert} per celluloid è in grado di convertire fra loro i formati di dati derivanti dal sequenziamento a singola cellula utilizzati da diversi algoritmi per la ricostruzione di alberi evoluzionari di tumori. Il tool può essere espanso aggiungendo altri formati a quelli supportati, aumentando ulteriormente l'utilità di celluloid nel ridurre le dimensioni dei dataset. L'utilizzo di PEG e packrat parser per la definizione e analisi rispettivamente dei formati permette infatti una facile aggiunta di nuovi formati, con minime modifiche del programma.
