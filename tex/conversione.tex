\section{similarità fra i formati}
SASC, SCITE e SPhyR utilizzano in input un dataset SCS, con piccole variazioni fra l' input richiesto dai tre programmi. infatti in tutti e tre i formati:
\begin{itemize}
\item il file è una matrice ternaria, che rappresenta la presenza di mutazioni nelle cellule
\item 1 rappresenta la presenza della mutazione
\item 0 rappresenta l'assenza della mutazione
\end{itemize}

\section{differenze fra i formati}
I tre formati di input indicano con un carattere diverso la mancanza di informazione per la combinazione di cellula e mutazione corrispondente, ovvero:
\begin{itemize}
\item 2 per SASC
\item 3 per SCITE
\item -1 per SPhyR
\end{itemize}

diversamente dagli altri due formati, SCITE utilizza le righe della matrice per rappresentare le mutazioni, e le colonne per rappresentare le cellule.

SPhyR richiede che le prime due righe del file comunichino il numero di cellule e mutazioni rispettivamente.

\section{grammatiche}
Il primo passo del programma per la conversione fra i formati è stato scrivere la grammatica dei formati sotto forma di PEG ed utilizzare TatSu per generare i parser. Ho scelto di ignorare gli spazi bianchi nel file di input, in quanto non cambiano le informazioni contenute nel file. 

Le grammatiche sono scritte nella sintassi di TatSu, che è una leggera variazione rispetto alla EBNF. Tatsu permette inoltre di aggiungere delle direttive nella grammatica, che controllano il comportamento del parser generato, e sono scritte nella forma \verb|@@name :: <value>|. in tutte e tre le grammatiche è presente la direttiva grammar, che specifica il nome della grammatica, e la direttiva whitespace, che definisce quali caratteri ignorare perchè spazi bianchi (in questo caso vengono ignorati gli spazi bianchi che non sono \textbackslash n).
Le grammatiche così ottenute sono le seguenti:

\subsection{SASC}
{\fontfamily{qcr}\selectfont
\begin{verbatim}
@@grammar :: SASC
@@whitespace ::  /(?s)[ \t\r\f\v]+/
start = file $ ;
cell = "0" ~ | "1" ~ | "2" ~ ;
row = {cell};
file = ("\n").{ row } ;
\end{verbatim}
}
\begin{itemize}
\item la regola \textit{start} matcha \textit{file} e il simbolo di \textit{end of text}, \$.
\item la regola \textit{file} matcha quante più \textit{row} possibili, suddivise da \textbackslash n.
\item la regola \textit{row} matcha quante più \textit{cell} possibili.
\item la regola \textit{cell} matcha 0, 1 o 2.
\end{itemize}

\subsection{SCITE}
{\fontfamily{qcr}\selectfont
\begin{verbatim}
@@grammar :: SCITE
@@whitespace ::  /(?s)[ \t\r\f\v]+/
start = file $ ;
cell = "0" ~ | "1" ~ | "2" ~ | "3" ~ ;
row = {cell};
file = ("\n").{ row } ;
\end{verbatim}
}
\begin{itemize}
\item la regola \textit{start} matcha \textit{file} e il simbolo di \textit{end of text}, \$.
\item la regola \textit{file} matcha quante più \textit{row} possibili, suddivise da \textbackslash n.
\item la regola \textit{row} matcha quante più \textit{cell} possibili.
\item la regola \textit{cell} matcha 0, 1, 2 o 3.
\end{itemize}

\subsection{SPhyR}
{\fontfamily{qcr}\selectfont
\begin{verbatim}
@@grammar :: SPHYR
@@whitespace ::  /(?s)[ \t\r\f\v]+/
@@eol_comments :: /#([^\n]*?)$/
start = cell_number_line "\n" snv_number_line "\n" file $ ;
cell_number_line = /\d+/;
snv_number_line = /\d+/;
cell = "0" ~ | "1" ~ | "-1" ~ ;
row = {cell};
file = ("\n").{row};
\end{verbatim}
}
Il formato di input SPhyR include commenti "python style", che iniziano con \# e terminano al primo carattere di newline. la direttiva @@eol\_comments definisce la struttura dei commenti che il parser deve ignorare.
\begin{itemize}
\item la regola \textit{start} matcha in \textit{cell\_number\_line} e \textit{snv\_number\_line} le prime due righe del file, ed in \textit{file} la matrice contenente i dati, assicurandosi che il file termini con il simbolo di \textit{end of text}, \$.
\item \textit{cell\_number\_line} e \textit{snv\_number\_line} contengono solo un numero.
\item la regola \textit{file} matcha quante più \textit{row} possibili, suddivise da \textbackslash n.
\item la regola \textit{row} matcha quante più \textit{cell} possibili.
\item la regola \textit{cell} matcha 0, 1, 2 o 3.
\end{itemize}

\section{codice}
La maggior parte del codice scritto si trova in \textbf{Translator.py}, che usa i moduli generati con TatSu \textbf{parserSASC}, \textbf{parserSCITE} e \textbf{parserSPHYR}.

\subsection{Translator}
\textbf{Translator.py} può essere eseguito individualmente o come parte di celluloid, e richiede i seguenti parametri da linea di comando:
\begin{itemize}
\item \textbf{inputFormat}, il formato del file da convertire (SASC,SCITE o SPhyR). Se omesso, il default è SASC.
\item \textbf{outputFormat}, il formato in cui si vuole convertire il file. Se omesso, il default è SASC.
\item \textbf{outfile}, il file di output. Se omesso, il default è stdout.
\item \textbf{file}, il file di input. 
\end{itemize}
\subsubsection{convert}
La funzione \textit{convert} prende come parametro un' oggetto argparse.Namespace contenente le variabili \textit{outputFormat}, \textit{inputFormat}, \textit{outfile} e \textit{file}, che rappresentano i rispettivi parametri da linea di comando descritti sopra.

La funzione legge il file di input, e successivamente chiama \textit{parse\_string}, \textit{translate} e \textit{write\_file} per analizzare, convertire e scrivere il file nel nuovo formato. 
\subsubsection{parse\_string}
La funzione \textit{parse\_string} ha due parametri:
\begin{itemize}
\item \textbf{file\_as\_string}, il contenuto del file di input sotto forma di stringa.
\item \textbf{file\_format}, il formato (SASC, SCITE, SPhyR) del file di input in una stringa.
\end{itemize}
L'output della funzione consiste in una lista di liste, dove ogni elemento delle liste più interne è un carattere della matrice contenuta nel file. 
\subsubsection{translate}
La funzione \textit{translate} ha tre parametri:
\begin{itemize}
\item \textbf{input\_ast}, la matrice di caratteri, sotto forma di lista di liste di caratteri. 
\item \textbf{format1}, il formato di partenza.
\item \textbf{format2}, il formato in cui si vuole tradurre
\end{itemize}
L'output della funzione è una lista di liste, dove ogni elemento delle liste più interne è un carattere della matrice nel formato di destinazione. 
\subsubsection{write\_file}
La funzione \textit{write\_file} ha tre parametri:
\begin{itemize}
\item \textbf{ast\_translated}, una lista di liste, contenente i caratteri da scrivere nel nuovo file.
\item \textbf{file\_name}, il nome del file in cui scrivere la matrice.
\item \textbf{file\_format}, il formato del file in cui scrivere la matrice.
\end{itemize}
La funzione scrive nel file scelto la matrice in input nel formato specificato.



























